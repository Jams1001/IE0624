\section{Conclusiones y Recomendaciones}
\begin{itemize}

\item  Se concluye que el mcu integrado en el arduino nano 33 BLE Sense es una poderosa herramienta para implementar modelos de inteligencia artificial y redes neoronales. Sus perifericos y funciones adicionales como que se puede programar con el lenguaje de programación Python a través de OpenMV IDE, son pluses que hacen una gran herramientaa para proyectos serios y de altura para proyectos con AI. Sumado a que el IDE que utiliza es de mayor comodidad con respecto a su antiguo predecesor.
\item Se recomienda la utilización de Google Colab Notebooks para agilizar la generación del modelo con el fin de aprovechar los recursos existentes y no consumir la capacidad de los recursos de los practicantes. Esto debido a que en primera instancia se tuvo problemas por la tardanza en la generación del modelo, la capacidad de memoria de las computadoras locales, entre otras cosas.
\item Se recomienda la futura implementación del mismo proyecto pero implementado con un STM32 para obtener el panorama completo y comparar las dificultades en la generación de modelos para diferentes microcontroladores.
\end{itemize}