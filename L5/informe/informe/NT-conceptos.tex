\subsection{Conceptos}

\subsubsection{AI}

La inteligencia artificial ( o por sus siglas en inglés \textit{«AI»}), la capacidad de una computadora digital o un robot controlado por computadora para realizar tareas comúnmente asociadas con seres inteligentes. El término se aplica con frecuencia al proyecto de desarrollar sistemas dotados de los procesos intelectuales característicos de los humanos, como la capacidad de razonar, descubrir significado, generalizar o aprender de experiencias pasadas \cite{AI}.

\subsubsection{Machine learning en microcontroladores}

El aprendizaje automático, mejor conocido como \textit{machine learning}, es una rama de la inteligencia artificial dirigida al desarrollo de herramientas que permiten que las máquinas aprendan sin estar explícitamente programadas para ello. Con dichas herramientas, se puede hacer que distintos dispositivos sean capaces identificar patrones entre un conjunto de datos, y a partir de ellos, realizar predicciones. El \textit{machine learning} puede utilizarse para crear tecnologías inteligentes que faciliten la vida de sus usuarios, como Google Assistant, para dar un ejemplo \cite{mnatraj}.

Sin embargo, la aplicación del \textit{machine learning} a menudo requiere de una gran cantidad de recursos, que pueden incluir un potente servidor en la nube o un computador con una considerable capacidad de procesamiento. Afortunadamente, ahora es posible ejecutar la inferencia de aprendizaje automático en hardware diminuto y de baja potencia, como los microcontroladores. Al traer el \textit{machine learning} a estos dispositivos se puede potenciar la inteligencia de miles de millones de equipos que se utilizan cotidianamente, y sin depender de hardware costoso o conexiones robustas a Internet \cite{mnatraj}. 

Una de las herramientas más populares para la implementación de \textit{machine learning} es TensorFlow. Este es un framework de aprendizaje automático, de código abierto y de Google para entrenar y ejecutar modelos de reconocimiento de patrones. Como parte de los esfuerzos de TensorFlow, Google también desarrolló una versión optimizada, destinada a ejecutar modelos de TensorFlow en microcontroladores. Dicha versión tiene el nombre de TensorFlow Lite For Microcontrollers. Este se adhiere a las restricciones requeridas en entornos embebidos, es decir, tiene un tamaño binario pequeño, no requiere el soporte de ningún sistema operativo, ninguna librería estándar de C o C++, o la asignación memoria dinámica \cite{mnatraj}.

\subsubsection{Red Neuronal artificial}

Consiste en una metodología de inteligencia artificial que enseña a las computadoras a procesar datos de una manera inspirada en el cerebro humano, con patrones o modelos previamente cargados de esos mismos eventos. Es un tipo de proceso de aprendizaje automático, llamado aprendizaje profundo, que utiliza nodos o neuronas interconectados en una estructura en capas que se asemeja al cerebro humano. Crea un sistema adaptativo que las computadoras usan para aprender de sus errores y mejorar continuamente. De esta forma las redes neuronales artificiales intentan resolver problemas complicados, como resumir documentos o reconocer rostros, reconocer movimientos, y todo esto en búsqueda de una mayor precisión para conquistar los límites de las capacidades humanas \cite{ANN}.





