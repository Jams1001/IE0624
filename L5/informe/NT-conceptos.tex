\subsection{Conceptos}
\subsubsection{AI}
La inteligencia artificial ( o por sus siglas en inglés \textit{«AI»}), la capacidad de una computadora digital o un robot controlado por computadora para realizar tareas comúnmente asociadas con seres inteligentes. El término se aplica con frecuencia al proyecto de desarrollar sistemas dotados de los procesos intelectuales característicos de los humanos, como la capacidad de razonar, descubrir significado, generalizar o aprender de experiencias pasadas \cite{AI}.
\subsubsection{Red Neuronal artificial}
Consiste en una metodología de inteligencia artificial que enseña a las computadoras a procesar datos de una manera inspirada en el cerebro humano, con patrones o modelos previamente cargados de esos mismos eventos. Es un tipo de proceso de aprendizaje automático, llamado aprendizaje profundo, que utiliza nodos o neuronas interconectados en una estructura en capas que se asemeja al cerebro humano. Crea un sistema adaptativo que las computadoras usan para aprender de sus errores y mejorar continuamente. De esta forma las redes neuronales artificiales intentan resolver problemas complicados, como resumir documentos o reconocer rostros, reconocer movimientos, y todo esto en búsqueda de una mayor precisión para conquistar los límites de las capacidades humanas \cite{ANN}.







