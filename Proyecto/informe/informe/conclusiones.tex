\section{Conclusiones y Recomendaciones}
\begin{itemize}
  \item Se recomienda exportar el script que muestrea las coordenadas a un modelo procesable por el microcontrolador. Esto a través de TensorFlow con una red neuronal entrenada para el reconocimiento y predicción de gestos manuales; esto para generar un sistema lo más autónomo posible, integrando una cámara adicional y sin la necesidad de un computador. Aprovechando la capacidad de un mcu para producir un sistema embebido. Esta potencial implementación probablemente implique exportar el proyecto a un microcontrolador más potente como el Arduino Nano BLE sense.
  \item El objetivo del proyecto se cumplió exitosamente. El controlador de luz es correcto y la comunicación serial de igual forma. Lo que deja al Arduino UNO como una plataforma que  le permite a los usuarios, no necesariamente únicamente del campo técnico-científico, una manera simple para crear objetos interactivos que pueden tener entradas de interruptores y sensores, para controlar salidas tanto físicas como digitales.
\item Se recomienda adaptar otro entorno de desarrollo integrado para programar y compilar “Arduino”, pues aunque el utilizado está basado en C/C++, en primera instancia solo permite trabajarse a sí mismo  bajo su propio IDE; el cual es bastante limitado. Por esto, a través de por ejemplo Visual estudio code, se pueden descargar extensiones e instalar librerías para utilizar un entorno más profesional.
\end{itemize}