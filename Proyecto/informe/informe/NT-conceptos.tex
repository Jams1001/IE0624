\subsection{Conceptos}

\subsubsection{IoT}
Por sus siglas del inglés \textit{«Internet of Things»}, el Internet de las cosas, si de alguna forma se puede englobar en una definición, consiste en la interconexión de objetos por medio de una red, generalmente inalámbrica, donde exista una interacción entre los mismos sin la necesaria invervención del factor humano.\\
Podría tratarse de sensores, calzado, vestuario, o hasta un lápiz! Esta idea involucra comunicación entre dos máquinas o M2M (machine to machine) y por ende, machine learning. Al utilizar un microcontrolador que se conecte a internet y que cuando exista un estímulo de movimiento, como por ejemplo un sismo, envíe sus aceleraciones reales en forma de datos a internet para ser visualizadas por otro dispositivo en cualquier otra parte del mundo, se hace uso de IoT.
 
\subsubsection{AI}

La inteligencia artificial ( o por sus siglas en inglés \textit{«AI»}), la capacidad de una computadora digital o un robot controlado por computadora para realizar tareas comúnmente asociadas con seres inteligentes. El término se aplica con frecuencia al proyecto de desarrollar sistemas dotados de los procesos intelectuales característicos de los humanos, como la capacidad de razonar, descubrir significado, generalizar o aprender de experiencias pasadas \cite{AI}.

\subsubsection{Red Neuronal artificial}

Consiste en una metodología de inteligencia artificial que enseña a las computadoras a procesar datos de una manera inspirada en el cerebro humano, con patrones o modelos previamente cargados de esos mismos eventos. Es un tipo de proceso de aprendizaje automático, llamado aprendizaje profundo, que utiliza nodos o neuronas interconectados en una estructura en capas que se asemeja al cerebro humano. Crea un sistema adaptativo que las computadoras usan para aprender de sus errores y mejorar continuamente. De esta forma las redes neuronales artificiales intentan resolver problemas complicados, como resumir documentos o reconocer rostros, reconocer movimientos, y todo esto en búsqueda de una mayor precisión para conquistar los límites de las capacidades humanas \cite{ANN}.

\subsubsection{Machine learning}

El aprendizaje automático, mejor conocido como \textit{machine learning}, es una rama de la inteligencia artificial dirigida al desarrollo de herramientas que permiten que las máquinas aprendan sin estar explícitamente programadas para ello. Con dichas herramientas, se puede hacer que distintos dispositivos sean capaces identificar patrones entre un conjunto de datos, y a partir de ellos, realizar predicciones. El \textit{machine learning} puede utilizarse para crear tecnologías inteligentes que faciliten la vida de sus usuarios, como Google Assistant, para dar un ejemplo \cite{mnatraj}.
