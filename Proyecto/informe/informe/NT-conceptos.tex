\subsection{Conceptos}

\subsubsection{IoT}
Por sus siglas del inglés \textit{«Internet of Things»}, el Internet de las cosas, si de alguna forma se puede englobar en una definición, consiste en la interconexión de objetos por medio de una red, generalmente inalámbrica, donde exista una interacción entre los mismos sin la necesaria invervención del factor humano.\\
Podría tratarse de sensores, calzado, vestuario, o hasta un lápiz! Esta idea involucra comunicación entre dos máquinas o M2M (machine to machine) y por ende, machine learning. Al utilizar un microcontrolador que se conecte a internet y que cuando exista un estímulo de movimiento, como por ejemplo un sismo, envíe sus aceleraciones reales en forma de datos a internet para ser visualizadas por otro dispositivo en cualquier otra parte del mundo, se hace uso de IoT.
 