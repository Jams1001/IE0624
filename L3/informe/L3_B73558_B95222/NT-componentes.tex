\subsection{Componentes complementarios}
\subsection{Componentes complementarios}
Para poder completar el proyecto se necesitó hacer uso de componentes adicionales además del MCU ATmega328P. Los mismos se detallan a continuación. Estos diseños están basados en electrónica para una parte de laboratorio, no para una implementación de uso real. 

\begin{itemize}
    \item Resistencia $100\Omega$ (x)
    \item Diodo LED (2)
\end{itemize}

A continuación se presenta una tabla con los precios en promedio en el mercado para estos componentes. No tiene sentido especificar muy meticulosamente estos precios pues son muy variables dependiendo el tiempo y los ofertantes, sin embargo, sí tiene sentido dar una aproximación general del potencial costo probable para implementar este diseño. Estos precios no consideran envío ni impuestos.


\begin{table}[H]
\centering
\begin{tabular}{|cc|c|}
\hline
\multicolumn{1}{|c|}{\textbf{Componente}} & \textbf{Cantidad} & \textbf{Precio (USD)} \\ \hline
\multicolumn{1}{|c|}{Botón pulsador}        & 2                 & 0.26             \\ \hline
\multicolumn{1}{|c|}{Resistencia $100\Omega$}        & 8                 & 2.72              \\ \hline
\multicolumn{1}{|c|}{Capacitor $10\mu F$}        & 1                 & 1.45              \\ \hline
\multicolumn{1}{|c|}{Diodo LED}        & 6                 & 3              \\ \hline
\multicolumn{1}{|c|}{MCU AT-tiny4313}        & 1                 & 1.68              \\ \hline
\multicolumn{2}{|c|}{\textbf{Total}}                          & \textbf{9.11}    \\ \hline
\end{tabular}
\label{componentes}
\caption{Lista de cantidad y precios de los componentes (Autoría propia).}
\end{table}