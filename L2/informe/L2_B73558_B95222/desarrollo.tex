\section{Desarrollo / Análisis de Resultados}
\subsection{Análisis de SW}
\begin{figure}[H]
    \centering
    \includegraphics[width=0.45\textwidth]{images/diagrama_asm.png}
    \caption{Diagrama ASM propuesto para el cruce de semáforos. Creación propia}
    \label{diagrama_asm}
\end{figure}

En la figura \ref{diagrama_asm} se puede observar el diagrama ASM construido para el cruce de semáforos mostrado en la figura \ref{cruce_semaforos}. En la siguiente tabla se presenta la descripción de la nomenclatura empleada en el diseño del diagrama: 

\begin{table}[H]
\centering
\renewcommand{\arraystretch}{1.25}
\begin{tabular}{|c|c|c|}
\hline
\multirow{5}{*}{Entradas} & B1   & Botón de semáforo peatonal 1                            \\ \cline{2-3} 
                          & B2   & Botón de semáforo peatonal 2                            \\ \cline{2-3} 
                          & TVP  & Timer de paso de vehículos o peatones (10 segundos)     \\ \cline{2-3} 
                          & TI   & Timer de 1 segundo                                      \\ \cline{2-3} 
                          & C3   & Contador de parpadeos (cuenta hasta 3 segundos)         \\ \hline
\multirow{4}{*}{Salidas}  & LDPV & Led de paso de vehículos                                \\ \cline{2-3} 
                          & LDPD & Led de peatones detenidos                               \\ \cline{2-3} 
                          & LDVD & Led de vehículos detenidos                              \\ \cline{2-3} 
                          & LDPP & Led de paso de peatones                                 \\ \hline
\multirow{5}{*}{Estados}  & A    & Paso de vehículos por al menos 10 segundos              \\ \cline{2-3} 
                          & B    & Parpadeo de luz verde del semáforo vehicular (LDPV)     \\ \cline{2-3} 
                          & C    & Paro total de vehículos y peatones por 1 segundo        \\ \cline{2-3} 
                          & D    & Paso de peatones por 10 segundos                        \\ \cline{2-3} 
                          & E    & Parpadeo de luces verdes de semáforos peatonales (LDPP) \\ \hline
\end{tabular}
\caption{Entradas, salidas y estados del diagrama ASM propuesto. Creación propia}
\renewcommand{\arraystretch}{1}
\end{table}

En programación orientada a objetos, una máquina de estados podría considerarse como una clase con una colección de variables de estado y métodos que activan las salidas correspondientes a cada estado. Sin embargo, en C no se tiene la posibilidad de escribir clases, ya que este es un lenguaje es procedimental. No obstante, hay formas de simular mecanismos propios de la programación orientada a objetos en C. En esta caso para imitar una clase se utilizó un \textit{struct}, una estructura especial de C, en la cual se definieron los miembros que componen cada estado de la máquina:

\begin{minted}{C}
typedef struct Semaforo{
  void (*state_func_ptr)(void);
  int time;
} FSM;
\end{minted}

El puntero \textit{state\_func\_ptr} apunta a la dirección de memoria del microcontrolador en la cual se almacena cada función. Este sirve para hacer los llamados a las funciones correspondientes de cada estado. Por otra parte, el atributo de \textit{time} es un número entero que determina la cantidad de segundos que cada estado debe permanecer activo.

Posteriormente se define la máquina de estados, haciendo un arreglo de 5 \textit{struct FSM}, uno por cada estado:

\begin{minted}{C}
FSM semaforo[5] = {
  {&A_out, 10},
  {&B_out, 3},
  {&C_out, 1},
  {&D_out, 10},
  {&E_out, 3},
};
\end{minted}

Como se puede observar en la porción de código mostrada anteriormente, cada espacio del arreglo \textit{semaforo} contiene un número entero, que representa el argumento \textit{time}. Además, en cada espacio se incluye el nombre de una de las siguientes funciones acompañado con el operador \&, para obtener la dirección en memoria de esa función. \textbf{(Esta mini-fracción de código solo es por comodidad y para fines ilustrativos).}

\begin{minted}{C}
void A_out(void){
  PORTB = 0x09;
}

void B_out(void){
  PORTB ^= 0x01;
}

void C_out(void){
  PORTB = 0x0A;
}

void D_out(void){
  PORTB = 0x06;
}

void E_out(void){
  PORTB ^= 0x04;
}
\end{minted}

Las funciones anteriormente expuestas se encargan de cambiar el estado de los pines del puerto B, según corresponda. Dicho esto, cabe destacar que las funciones \textit{B\_out()} y \textit{E\_out()} invierten el estado de los pines B0 y B2, debido a que estas son las funciones que generan los parpadeos de luces.

Por otro lado, la configuración del timer se encapsuló dentro de la siguiente función:

\begin{minted}{C}
void timer_setup() {
  TCCR0A=0x00; // Modo normal
  TCCR0B=0x00; 
  TCCR0B |= (1<<CS00)|(1<<CS02); // Prescaling de 1024
  sei(); // Se llama a la función sei() para habilitar las interrupciones globales
  TCNT0=0;
  TIMSK|=(1<<TOIE0); // Se habilita la interrupción del timer1
}
\end{minted}

\newpage

Tomando en cuenta que el ATtiny4313 opera con un oscilador interno a $8\,MHz$ y que el prescaling establecido fue de el de 1024, se puede determinar cuantos ciclos del timer se necesitan para contar un segundo de la siguiente forma:

\begin{equation*}
    \text{Frecuencia con prescaling} = \frac{8\,MHz}{1024} = 7812.5 \,Hz
\end{equation*}

\begin{equation*}
    \text{Periodo con prescaling} = \frac{1}{7812.5 \,Hz} = 0.128\,ms 
\end{equation*}

Para terminar una cuenta del timer se necesitan entonces:

\begin{equation*}
    255 \cdot 0.128\,ms = 0.03264\,s
\end{equation*}

Por lo tanto, para contar un segundo se necesitan:

\begin{equation*}
    \frac{1\,s}{0.03264\,s} \approx \text{31 ciclos}
\end{equation*}

Cabe mencionar que el resultado anterior se bajó a $30\,s$ por comodidad.

Más adelante, se configuró el Interrupt service routine de la siguiente forma, para que únicamente cambiara el estado de una variable \textit{B1\_B2}, utilizada para guardar la solicitud de paso peatonal:

\begin{minted}{C}
ISR (INT1_vect) {     
  B1_B2 = 1;
}
\end{minted}

Por otro lado, el Interrupt vector para el Timer0 se configuró de la siguiente forma:

\begin{minted}{C}
ISR (TIMER0_OVF_vect){
  if((int_count) == 1 || (int_count == 15)){ // Cuenta de medio segundo
    if(state == B){
      (semaforo[B].state_func_ptr)(); // Llamado a salidas de estado B
    }
    if(state == E){
      (semaforo[E].state_func_ptr)(); // Llamado a salidas de estado B
    }
  }
  else if(int_count == 30){ // cuenta de un segundo
    ++TI;
    int_count = 0;
  }
  if(TI == 10){
    ++TVP;
  }
  ++int_count;
}
\end{minted}

Aquí se puede observar que hizo uso de la cuenta de 30 ciclos por segundo para determinar el aumento de las variables de temporización \textit{TI},\textit{TVP} y \textit{int\_count}. La última de estas aumenta con cada ciclo y, por lo tanto, al llegar a 30 cuenta un segundo. Asimismo, es en esta parte que se llama a las funciones \textit{B\_out()} y \textit{E\_out()}, las cuales se encargan de hacer que las luces parpadeen. Ambas se llaman cada medio segundo durante un total de 3 segundos.

Finalmente, en la función \textit{main()} primeramente se hacen las configuraciones necesarias para que el microcontrolador opere de la manera deseada, posteriormente se hace la inicialización de todas las variables y por último se ejecuta un bucle infinito dentro del cual corre la máquina de estados desarrollada.

\begin{minted}{C}
int main(void){
  DDRB = 0x0F; // Configurando los pines de entrada/salida del puerto B
  GIMSK = 0x80; // Habilitando la interrupción externa en INT1  
  MCUCR = 0x08; // Interrupción generada por el flanco decreciente en INT1
  PORTB = 0x00; // Se setean todas las salidas en estado bajo (y se activan las resistencias de pull-up de todas las entradas)
  timer_setup(); // Llamado a configuración del timer

  state = A;
  B1_B2 = 0;
  TVP = 0;
  TI = 0;
  int_count = 0;
  pass_flag = 0;

  while(1){
    switch (state){
      case A:
        .
        .
      case B:
        .
        .
      case C:
        .
        .
      case D:
        .
        .
      case E:
        .
        .
    }
  }
}
\end{minted}

\subsection{Análisis de HW}

A continuación se analiza el esquemático y sus componentes con osciloscopios y multimetros para la verificación de su correcto funcionamiento.


Para poder visualizar el comportamiento de las señales se ajustó la escala del osciloscopio a $1s$. A continuación se presenta un fragmento de la señal del osciloscopio (lo más que se fue capaz de extender) la cual coincide con  el diagram de tiempo solicitado en figura \ref{diagrama_temporizacion}. La siguiente figura se encuentra en el estado D y muesta el estado A, B, C y D.

\begin{figure}[H]
\centering
\includegraphics[scale=0.5]{./images/os.png} 
\caption{Framento de diagrama de tiempo.}
\label{f1}
\end{figure}

\begin{figure}[H]
\centering
\includegraphics[scale=0.8]{./images/A.png} 
\caption{Circuito en estado A.}
\label{f1}
\end{figure}


\begin{figure}[H]
\centering
\includegraphics[scale=0.8]{./images/A1.png} 
\caption{Circuito en estado A justo en donde ocurre la interrupción o primera entrada.}
\label{f1}
\end{figure}

En la siguiente figura el LED encerrado en un cuadro verde está haciendo blink.
\begin{figure}[H]
\centering
\includegraphics[scale=0.76]{./images/B.png} 
\caption{Circuito en estado B.}
\label{f1}
\end{figure}


\begin{figure}[H]
\centering
\includegraphics[scale=0.76]{./images/C.png} 
\caption{Circuito en estado C.}
\label{f1}
\end{figure}


\begin{figure}[H]
\centering
\includegraphics[scale=0.76]{./images/D.png} 
\caption{Circuito en estado D.}
\label{f1}
\end{figure}


En la siguiente figura los LEDs encerrados en un cuadro verde están haciendo blink.
\begin{figure}[H]
\centering
\includegraphics[scale=0.76]{./images/E.png} 
\caption{Circuito en estado E.}
\label{f1}
\end{figure}










