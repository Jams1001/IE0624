\section{Conclusiones y Recomendaciones}
\begin{itemize}
\item Se concluye que las técnicas de interrupción son un método más elegante, eficiente, y rápido para realizar cambios de estado en un sistema electrónico en comparación con lógica de HW; la cual podría depender de inestabilidades en las señales. Lo mejor es aprovechar al máximo las posibilidades que ofrece el microcontrolador pues a escala de diseño digital es mucho menor la probabilidad de errores. 
\item Se mejoró el aprendizaje general del manejo de rutinas de interrupción y temporizadores en firmware para el ATtiny4313; aprendizaje que podría extenderse a diferentes MCUs.
\item Como recomendación se presenta que la lógica de las rutinas de interrupción debe ser muy corta y directa. \item Adicionalmente los practicantes experimentaron que la investigación previa al diseño es vital para un buen desarrollo. No importa si esta toma la mayor parte del tiempo, partir de cero en en este tipo de proyectos es absolutamente en vano. Es por esto que se recomienda una previa investigación y lectura, con calma, de la hoja de datos del fabricante.
\end{itemize}